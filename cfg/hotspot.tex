\documentclass[a4paper]{article}

\usepackage{graphicx}
\usepackage{float}
\usepackage{hyperref}
\usepackage{enumitem}
\usepackage[utf8]{inputenc}
\usepackage{tocloft}
\usepackage{multicol}

\usepackage[margin=0.5in]{geometry}

\setlength\cftbeforesecskip{1pt}
\setlength\cftaftertoctitleskip{2pt}

\setlist[enumerate]{noitemsep}
\setlist[itemize]{noitemsep}

\begin{document}

\title{{\tt @@PROGRAM@@} Performance Report}

\date{@@TIMESTAMP@@}
\author{}

\maketitle

\abstract{This performance report is intended to support performance analysis and optimization activities. It includes details on program behavior, system configuration and capabilities, and of course relevant data from well-known performance analysis tools. This report was generated using {\tt hotspot} version {\tt 0.1}. Homepage \url{http://www.github.com/moreandres/hotspot}. Please report issues to \url{mailto:more.andres@gmail.com}. Full Log: \url{~/.hotspot/@@PROGRAM@@/@@TIMESTAMP@@/hotspot.log}.}

\begin{multicols}{2}
  \tableofcontents
\end{multicols}

\section{Program}

This section provides details about the program being analyzed.

\begin{enumerate}
\item Program: {\tt @@PROGRAM@@}
\item Timestamp: {\tt @@TIMESTAMP@@}
\item Parameters Range: {\tt @@RANGE@@}
\end{enumerate}

Program is the name of the program binary.
Timestamp is a unique identifier used to store information on disk.
Parameters Range is the problem size set used to scale the program. 

\section{System Capabilities}

This section provides details about the system being used for the analysis.

\subsection{System Configuration}

This subsection provides details about the system configuration.

The hardware in the system is summarized using a hardware lister utility.

\begin{verbatim}
@@HARDWARE@@
\end{verbatim}

The software in the system is summarized using the GNU/Linux platform.

\begin{verbatim}
@@PLATFORM@@
\end{verbatim}

The software toolchain is also detailed.

\begin{enumerate}
\item Host: {\tt @@HOST@@}
\item Distribution: {\tt @@DISTRO@@}
\item Compiler: {\tt @@COMPILER@@}
\item C Library: {\tt @@LIBC@@}
\end{enumerate}

\subsection{System Performance Baseline}

This subsection provides details about the system capabilities.

A set of performance benchmark results is included as a reference.

\begin{table}[H]
\caption{Benchmarks}
  \centering
    \begin{tabular}{|l|l|l|}\hline
      {\bf Benchmark} & {\bf Value} & {\bf Unit} \\ \hline
      hpl & @@HPCC-HPL@@ & tflops \\ \hline
      dgemm & @@HPCC-DGEMM@@ & mflops \\ \hline
      ptrans & @@HPCC-PTRANS@@ & MB/s \\ \hline
      random & @@HPCC-RANDOM@@ & MB/s \\ \hline
      stream & @@HPCC-STREAM@@ & MB/s \\ \hline
      fft & @@HPCC-FFT@@ & MB/s \\ \hline
    \end{tabular}
 \label{table:pruebas}
\end{table}

\section{Workload}

This section provides details about the workload behavior.

\begin{enumerate}
\item Execution time:
\begin{enumerate}
\item problem size: {\tt @@FIRST@@ - @@LAST@@} seconds
\item geomean: {\tt @@GEOMEAN@@} seconds
\item average: {\tt @@AVERAGE@@} seconds
\item stddev: {\tt @@STDDEV@@}
\item min: {\tt @@MIN@@} seconds
\item max: {\tt @@MAX@@} seconds
\item repetitions: {\tt @@COUNT@@} times
%\item outliers: {\tt @@OUTLIERS@@}
\end{enumerate}

\begin{figure}[H]
\label{fig:histogram}
\centering
\includegraphics[width=\textwidth]{hist.pdf}
\caption{Results Distribution}
\end{figure}

\end{enumerate}

\section{Scalability}

This section provides details about the scaling behavior of the program.

A chart with the execution time when scaling the problem size.

\begin{figure}[H]
\label{fig:scaling}
\centering
\includegraphics[width=\textwidth]{data.pdf}
\caption{Problem size times}
\end{figure}

A chart with the execution time when scaling computation units.

\begin{figure}[H]
\label{fig:threads}
\centering
\includegraphics[width=\textwidth]{procs.pdf}
\caption{Thread count times}
\end{figure}

The parallel and serial fractions of the program can be estimated.

\begin{enumerate}
\item Parallel Fraction: {\tt @@PARALLEL@@}
\item Serial: {\tt @@SERIAL@@}
\end{enumerate}

Optimization limits can be estimated using scaling laws.

\begin{enumerate}
\item Amdalah Law for 1024 procs: {\tt @@AMDALAH@@ times}
\item Gustafson Law for 1024 procs: {\tt @@GUSTAFSON@@ times}
\end{enumerate}

\section{Profile}

This section provides details about the execution profile of the program and the system.

\subsection{Program Profiling}

This subsection provides details about the program execution profile.

\subsubsection{Call Graph}

\begin{verbatim}
92.6% main -> fft -> _fft -> _muldc3
\end{verbatim}

\subsubsection{Flat Profile}

\begin{verbatim}
@@PROFILE@@
\end{verbatim}

\subsection{System Profiling}

This subsection provide details about the system execution profile.

\subsubsection{System Resources Usage}

\begin{figure}[H]
\label{fig:cpu}
\centering
\includegraphics[width=\textwidth]{CPU.pdf}
\caption{CPU Usage}
\end{figure}

\begin{figure}[H]
\label{fig:memory}
\centering
\includegraphics[width=\textwidth]{MEM.pdf}
\caption{Memory Usage}
\end{figure}

\begin{figure}[H]
\label{fig:reads}
\centering
\includegraphics[width=\textwidth]{kBrds.pdf}
\caption{Reads from Disk}
\end{figure}

\begin{figure}[H]
\label{fig:writes}
\centering
\includegraphics[width=\textwidth]{kBwrs.pdf}
\caption{Writes to Disk}
\end{figure}

\subsubsection{Bottlenecks}

\begin{verbatim}
@@ANNOTATION@@
\end{verbatim}

\section{Low Level}

This section provide details about low level details such as vectorization and performance counters.

\subsection{Vectorization Report}

This subsection provide details about vectorization status of the program loops.

\begin{verbatim}
@@VECTORIZER@@
\end{verbatim}

\subsection{Counters Report}

This subsection provides details about software and hardware counters.

\begin{verbatim}
@@COUNTERS@@
\end{verbatim}

\end{document}
