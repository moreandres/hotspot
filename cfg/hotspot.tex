\documentclass[a4paper]{article}

\usepackage{graphicx}
\usepackage{float}
\usepackage{hyperref}
\usepackage{enumitem}
\usepackage[utf8]{inputenc}
\usepackage{tocloft}
\usepackage{multicol}

\usepackage[margin=0.5in]{geometry}

\setlength\cftbeforesecskip{1pt}
\setlength\cftaftertoctitleskip{2pt}

\setlist[enumerate]{noitemsep}
\setlist[itemize]{noitemsep}

\pagenumbering{arabic}

\begin{document}

\title{{\tt @@PROGRAM@@} Performance Report}

\date{@@TIMESTAMP@@}
\author{}

\maketitle

\abstract{This performance report is intended to support performance analysis and optimization activities. It includes details on program behavior, system configuration and capabilities, including support data from well-known performance analysis tools.

% \cite{openmp}

\smallskip

This report was generated using {\tt hotspot} version {\tt 0.1}. Homepage \url{http://www.github.com/moreandres/hotspot}. Full execution log can be found at \url{~/.hotspot/@@PROGRAM@@/@@TIMESTAMP@@/hotspot.log}.}

\begin{multicols}{2}
  \tableofcontents
\end{multicols}

\section{Program}

This section provides details about the program being analyzed.

\begin{enumerate}
\item Program: {\tt @@PROGRAM@@}. \\Program is the name of the program.
\item Timestamp: {\tt @@TIMESTAMP@@}. \\Timestamp is a unique identifier used to store information on disk.
\item Parameters Range: {\tt @@RANGE@@}. \\Parameters range is the problem size set used to scale the program. 
\end{enumerate}

\section{System Capabilities}

This section provides details about the system being used for the analysis.

\subsection{System Configuration}

This subsection provides details about the system configuration.

The hardware in the system is summarized using a hardware lister utility.
It reports exact memory configuration, firmware version, mainboard configuration, CPU version and speed, cache configuration, bus speed and others.

\smallskip

The hardware configuration can be used to contrast the system capabilities to well-known benchmarks results on similar systems.

\begin{verbatim}
@@HARDWARE@@
\end{verbatim}

The software in the system is summarized using the GNU/Linux platform string.

\begin{verbatim}
@@PLATFORM@@
\end{verbatim}

The software toolchain is built upon the following components.

\begin{enumerate}
\item Host: {\tt @@HOST@@}
\item Distribution: {\tt @@DISTRO@@}. \\This codename provides LSB (Linux Standard Base) and distribution-specific information.
\item Compiler: {\tt @@COMPILER@@}. \\Version number of the compiler program.
\item C Library: {\tt @@LIBC@@}. \\Version number of the C library.
\end{enumerate}

\smallskip

The software configuration can be used to contrast the system capabilities to well-known benchmark results on similar systems.

\subsection{System Performance Baseline}

This subsection provides details about the system capabilities.

\smallskip

A set of performance results is included as a reference to contrast systems and to verify hardware capabilities using well-known synthetic benchmarks.

\smallskip

The HPC Challenge benchmark \cite{hpcc} consists of different tests:

\begin{enumerate}
\item{HPL}: the Linpack TPP benchmark which measures the floating point rate of execution for solving a linear system of equations.
\item{DGEMM}: measures the floating point rate of execution of double precision real matrix-matrix multiplication.
\item{PTRANS (parallel matrix transpose)}: exercises the communications where pairs of processors communicate with each other simultaneously.
\item{RandomAccess}: measures the rate of integer random updates of memory (GUPS).
\item{STREAM}: a simple synthetic benchmark program that measures sustainable memory bandwidth (in GB/s).
\item{FFT}: measures the floating point rate of execution of double precision complex one-dimensional Discrete Fourier Transform (DFT).
\end{enumerate}

\begin{table}[H]
\caption{Benchmarks}
  \centering
    \begin{tabular}{|l|l|l|}\hline
      {\bf Benchmark} & {\bf Value} & {\bf Unit} \\ \hline
      hpl & @@HPCC-HPL@@ & tflops \\ \hline
      dgemm & @@HPCC-DGEMM@@ & mflops \\ \hline
      ptrans & @@HPCC-PTRANS@@ & MB/s \\ \hline
      random & @@HPCC-RANDOM@@ & MB/s \\ \hline
      stream & @@HPCC-STREAM@@ & MB/s \\ \hline
      fft & @@HPCC-FFT@@ & MB/s \\ \hline
    \end{tabular}
 \label{table:pruebas}
\end{table}

\smallskip

Most programs will have a dominant compute kernel that can be approximated by the ones above, the results helps to understand the available capacity.

\section{Workload}

This section provides details about the workload behavior.

\subsection{Workload Stability}

This subsection provides details about workload stability.

\begin{enumerate}
\item Execution time:
\begin{enumerate}
\item problem size range: {\tt @@FIRST@@ - @@LAST@@}
\item geomean: {\tt @@GEOMEAN@@} seconds
\item average: {\tt @@AVERAGE@@} seconds
\item stddev: {\tt @@STDDEV@@}
\item min: {\tt @@MIN@@} seconds
\item max: {\tt @@MAX@@} seconds
\item repetitions: {\tt @@COUNT@@} times
\end{enumerate}
\end{enumerate}

The histogram plots the elapsed times and shows how they fit in a normal distribution sample.

\begin{figure}[H]
\label{fig:histogram}
\centering
\includegraphics[width=\textwidth]{hist.pdf}
\caption{Results Distribution}
\end{figure}

The workload should run for at least one minute to fully utilize system resources. The execution time of the workload should be stable and the standard deviation less than 3 units.

\subsection{Workload Optimization}

This section shows how the program reacts to different optimization levels.

\begin{figure}[H]
\label{fig:optimizations}
\centering
\includegraphics[width=\textwidth]{opts.pdf}
\caption{Optimization Levels}
\end{figure}

\section{Scalability}

This section provides details about the scaling behavior of the program.

\subsection{Problem Size Scalability}

A chart with the execution time when scaling the problem size.

\begin{figure}[H]
\label{fig:scaling}
\centering
\includegraphics[width=\textwidth]{data.pdf}
\caption{Problem size times}
\end{figure}

The chart will show how computing time increases when increasing problem size.
There should be no valleys or bumps if processing properly balanced across computational units.

\subsection{Computing Scalability}

A chart with the execution time when scaling computation units.

\begin{figure}[H]
\label{fig:threads}
\centering
\includegraphics[width=\textwidth]{procs.pdf}
\caption{Thread count times}
\end{figure}

The chart will show how computing time decreases when increasing processing units.
An ideal scaling line is provided for comparison.

The parallel and serial fractions of the program can be estimated using the information above.

\begin{enumerate}
\item Parallel Fraction: {\tt @@PARALLEL@@}.\\Portion of the program doing parallel work.
\item Serial: {\tt @@SERIAL@@}.\\Portion of the program doing serial work.
\end{enumerate}

Optimization limits can be estimated using scaling laws.

\begin{enumerate}
\item Amdalah Law for 1024 procs: {\tt @@AMDALAH@@ times}.\\ Optimizations are limited up to this point when scaling problem size. \cite{amdalah}
\item Gustafson Law for 1024 procs: {\tt @@GUSTAFSON@@ times}.\\ Optimizations are limited up to this point when not scaling problem size. \cite{gustafson}
\end{enumerate}

\section{Profile}

This section provides details about the execution profile of the program and the system.

\subsection{Program Profiling}

This subsection provides details about the program execution profile.

% TODO: call graph?

\subsubsection{Flat Profile}

The flat profile shows how much time your program spent in each function, and how many times that function was called.

\begin{verbatim}
@@PROFILE@@
\end{verbatim}

The table shows where to focus optimization efforts to maximize impact.

\subsection{System Profiling}

This subsection provide details about the system execution profile.

\subsubsection{System Resources Usage}

The following charts shows the state of system resources during the execution of the program.

\begin{figure}[H]
\label{fig:cpu}
\centering
\includegraphics[width=\textwidth]{CPU.pdf}
\caption{CPU Usage}
\end{figure}

Note that this chart is likely to show as upper limit a multiple of 100\% in case a multicore system is being used.

\begin{figure}[H]
\label{fig:memory}
\centering
\includegraphics[width=\textwidth]{MEM.pdf}
\caption{Memory Usage}
\end{figure}

\begin{figure}[H]
\label{fig:reads}
\centering
\includegraphics[width=\textwidth]{kBrds.pdf}
\caption{Reads from Disk}
\end{figure}

\begin{figure}[H]
\label{fig:writes}
\centering
\includegraphics[width=\textwidth]{kBwrs.pdf}
\caption{Writes to Disk}
\end{figure}

\subsection{Hotspots}

This subsection shows annotated code guiding the optimization efforts.

\begin{verbatim}
@@ANNOTATION@@
\end{verbatim}

\section{Low Level}

This section provide details about low level details such as vectorization and performance counters.

\subsection{Vectorization Report}

This subsection provide details about vectorization status of the program loops.

\begin{verbatim}
@@VECTORIZER@@
\end{verbatim}

The details above shows the list of loops in the program and if they are being vectorized or not.
These reports can pinpoint areas where the compiler cannot apply vectorization and related optimizations.
It may be possible to modify your code or communicate additional information to the compiler to guide the vectorization and/or optimizations.

\subsection{Counters Report}

This subsection provides details about software and hardware counters.

\begin{verbatim}
@@COUNTERS@@
\end{verbatim}

The details above shows counters that provide low-overhead access to detailed performance information using internal registers of the CPU.

\begin{thebibliography}{9}

\bibitem{hpcc}
  {Piotr Luszczek and Jack J. Dongarra and David Koester and Rolf Rabenseifner and Bob Lucas and Jeremy Kepner and John Mccalpin and David Bailey and Daisuke Takahashi},
  \emph{Introduction to the HPC Challenge Benchmark Suite}.
  Technical Report,
  2005.

\bibitem{amdahl}
  {Amdahl, Gene M.},
  \emph{Validity of the single processor approach to achieving large scale computing capabilities}.
  Communications of the ACM,
  {Proceedings of the April 18-20, 1967, spring joint computer conference Pages 483-485},
  1967.

\bibitem{gustafson}
  John L. Gustafson,
  \emph{Reevaluating Amdahl's Law}.
  Communications of the ACM,
  Volume 31 Pages 532-533,
  1988.

\bibitem{openmp}
  OpenMP Architecture Review Board,
  \emph{OpenMP Application Program Interface}.
  {\it http://www.openmp.org},
  3.0,
  May 2008.

\end{thebibliography}

\end{document}
